
%\documentclass[11pt]{article}
\documentclass[12pt]{scrartcl}

\usepackage[top=2.54cm, bottom=2.54cm, left=2.54cm, right=2.54cm]{geometry} 
\usepackage{stfloats}
\usepackage{pstool}
%\usepackage{subfigure}
\usepackage{amsmath}
\usepackage{titling}
%\usepackage{tabular}
\usepackage{gensymb}
\usepackage[font=small,labelfont=bf]{caption}
\usepackage{subcaption}
\usepackage{gensymb}
\usepackage[scientific-notation=true]{siunitx}
\setlength{\droptitle}{-2.5cm}

\begin{document}
\title{Random ramblings about units in firn models}
%\subtitle{Exercise 2: Fourier Transform of a Gaussian and Convolution}
\author{Max Stevens}
\maketitle

Rob Arthern used a steady-accumulation assumption to formulate a simplified version of the model. The Nabarro-Herring creep and grain-growth physics are coupled to find rate coefficients $c_0$ and $c_1$ for the firn-densification Equations:

\begin{equation}
\label{eq:firn_general}
\frac{D\rho}{Dt} = c_0(\rho_i -\rho) \quad \quad \rho\le550 \text{kg m}^{-3}
\end{equation}
\begin{equation}
\label{eq:firn_general2}
\frac{D\rho}{Dt} = c_1 (\rho_i -\rho) \quad \quad \rho>550  \text{kg m}^{-3}
\end{equation}

Rob uses the coefficients:

\begin{equation}
\label{eq:arthern}
c_0 =  0.07\, \dot{b}\,g \ \textrm{exp}\bigg[-\frac{E_c}{RT}+\frac{E_g}{RT_{av}}\bigg],
\end{equation}
\begin{equation}
\label{eq:arthern2}
c_1 = 0.03\, \dot{b}\,g \ \textrm{exp}\bigg[-\frac{E_c}{RT}+\frac{E_g}{RT_{av}}\bigg].
\end{equation}

The activation energies for creep $E_c$ and grain-growth $E_g$ are 60 kJ mol$^{-1}$ and 42.4 kJ mol$^{-1}$ and here $\dot{b}$ has units of \textbf{kg m$^{-2}$ a$^{-1}$}.

The Arrhenius term is unitless. Rob shows in his appendix that the units of the coefficients c$_{0}$ and c$_{1}$ are \textbf{ m s$^{2}$ kg$^{-1}$}, so the units are balanced on each side of the equation.\\

The Ligtenberg and Kuipers Munneke model multiply equations \ref{eq:arthern} and \ref{eq:arthern2} by M$_{0}$ and M$_{1}$:

\begin{equation}
M_0= 1.435-0.151\, \textrm{ln}(\dot{b})  \quad \quad \rho<550  \text{ kg m}^{-3},
\end{equation}
\begin{equation}
M_1 = 2.366-0.293\, \textrm{ln}(\dot{b})  \quad \quad \rho>550  \text{ kg m}^{-3}.
\end{equation}

so that their coefficients $c_0$ and $c_1$ to be plugged into equations \ref{eq:firn_general} and \ref{eq:firn_general2} are: 
\begin{equation}
c_0 = 0.07\,\dot{b}\,g \, [1.435-0.151\, \textrm{ln}(\dot{b})]  \: \textrm{exp}\bigg[-\frac{E_c}{RT}+\frac{E_g}{RT_{av}}\bigg]
\end{equation}
\begin{equation}
c_1 = 0.03\,\dot{b}\,g \, [2.366-0.293\, \textrm{ln}(\dot{b})]  \: \textrm{exp}\bigg[-\frac{E_c}{RT}+\frac{E_g}{RT_{av}}\bigg]
\end{equation}




But, they state that their units for $\dot{b}$ are \textbf{mm a$^{-1}$}. It turns out that numerically \textbf{mm a$^{-1}$} and \textbf{kg m$^{-2}$ a$^{-1}$} are the same. 

But, Ligtenberg does not mention the units of his coefficients at all. Since he is using the 0.03 and 0.07 from Rob, the units of M$_{0}$ and M$_{1}$ end up being \textbf{kg mm$^{-1}$ m$^{-2}$}, or $\mathbf{10^{3}}$ \textbf{kg m$^{-3}$} (I think I did that correctly!). That is ignoring the fact that they are taking the natural log of a quantity with units. So, for now I suggest not diving too deep into this rabbit hole :)





\end{document}





